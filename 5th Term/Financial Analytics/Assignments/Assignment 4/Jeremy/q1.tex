% Options for packages loaded elsewhere
\PassOptionsToPackage{unicode}{hyperref}
\PassOptionsToPackage{hyphens}{url}
%
\documentclass[
]{article}
\usepackage{lmodern}
\usepackage{amssymb,amsmath}
\usepackage{ifxetex,ifluatex}
\ifnum 0\ifxetex 1\fi\ifluatex 1\fi=0 % if pdftex
  \usepackage[T1]{fontenc}
  \usepackage[utf8]{inputenc}
  \usepackage{textcomp} % provide euro and other symbols
\else % if luatex or xetex
  \usepackage{unicode-math}
  \defaultfontfeatures{Scale=MatchLowercase}
  \defaultfontfeatures[\rmfamily]{Ligatures=TeX,Scale=1}
\fi
% Use upquote if available, for straight quotes in verbatim environments
\IfFileExists{upquote.sty}{\usepackage{upquote}}{}
\IfFileExists{microtype.sty}{% use microtype if available
  \usepackage[]{microtype}
  \UseMicrotypeSet[protrusion]{basicmath} % disable protrusion for tt fonts
}{}
\makeatletter
\@ifundefined{KOMAClassName}{% if non-KOMA class
  \IfFileExists{parskip.sty}{%
    \usepackage{parskip}
  }{% else
    \setlength{\parindent}{0pt}
    \setlength{\parskip}{6pt plus 2pt minus 1pt}}
}{% if KOMA class
  \KOMAoptions{parskip=half}}
\makeatother
\usepackage{xcolor}
\IfFileExists{xurl.sty}{\usepackage{xurl}}{} % add URL line breaks if available
\IfFileExists{bookmark.sty}{\usepackage{bookmark}}{\usepackage{hyperref}}
\hypersetup{
  pdftitle={Untitled},
  pdfauthor={Yijie Wu},
  hidelinks,
  pdfcreator={LaTeX via pandoc}}
\urlstyle{same} % disable monospaced font for URLs
\usepackage[margin=1in]{geometry}
\usepackage{color}
\usepackage{fancyvrb}
\newcommand{\VerbBar}{|}
\newcommand{\VERB}{\Verb[commandchars=\\\{\}]}
\DefineVerbatimEnvironment{Highlighting}{Verbatim}{commandchars=\\\{\}}
% Add ',fontsize=\small' for more characters per line
\usepackage{framed}
\definecolor{shadecolor}{RGB}{248,248,248}
\newenvironment{Shaded}{\begin{snugshade}}{\end{snugshade}}
\newcommand{\AlertTok}[1]{\textcolor[rgb]{0.94,0.16,0.16}{#1}}
\newcommand{\AnnotationTok}[1]{\textcolor[rgb]{0.56,0.35,0.01}{\textbf{\textit{#1}}}}
\newcommand{\AttributeTok}[1]{\textcolor[rgb]{0.77,0.63,0.00}{#1}}
\newcommand{\BaseNTok}[1]{\textcolor[rgb]{0.00,0.00,0.81}{#1}}
\newcommand{\BuiltInTok}[1]{#1}
\newcommand{\CharTok}[1]{\textcolor[rgb]{0.31,0.60,0.02}{#1}}
\newcommand{\CommentTok}[1]{\textcolor[rgb]{0.56,0.35,0.01}{\textit{#1}}}
\newcommand{\CommentVarTok}[1]{\textcolor[rgb]{0.56,0.35,0.01}{\textbf{\textit{#1}}}}
\newcommand{\ConstantTok}[1]{\textcolor[rgb]{0.00,0.00,0.00}{#1}}
\newcommand{\ControlFlowTok}[1]{\textcolor[rgb]{0.13,0.29,0.53}{\textbf{#1}}}
\newcommand{\DataTypeTok}[1]{\textcolor[rgb]{0.13,0.29,0.53}{#1}}
\newcommand{\DecValTok}[1]{\textcolor[rgb]{0.00,0.00,0.81}{#1}}
\newcommand{\DocumentationTok}[1]{\textcolor[rgb]{0.56,0.35,0.01}{\textbf{\textit{#1}}}}
\newcommand{\ErrorTok}[1]{\textcolor[rgb]{0.64,0.00,0.00}{\textbf{#1}}}
\newcommand{\ExtensionTok}[1]{#1}
\newcommand{\FloatTok}[1]{\textcolor[rgb]{0.00,0.00,0.81}{#1}}
\newcommand{\FunctionTok}[1]{\textcolor[rgb]{0.00,0.00,0.00}{#1}}
\newcommand{\ImportTok}[1]{#1}
\newcommand{\InformationTok}[1]{\textcolor[rgb]{0.56,0.35,0.01}{\textbf{\textit{#1}}}}
\newcommand{\KeywordTok}[1]{\textcolor[rgb]{0.13,0.29,0.53}{\textbf{#1}}}
\newcommand{\NormalTok}[1]{#1}
\newcommand{\OperatorTok}[1]{\textcolor[rgb]{0.81,0.36,0.00}{\textbf{#1}}}
\newcommand{\OtherTok}[1]{\textcolor[rgb]{0.56,0.35,0.01}{#1}}
\newcommand{\PreprocessorTok}[1]{\textcolor[rgb]{0.56,0.35,0.01}{\textit{#1}}}
\newcommand{\RegionMarkerTok}[1]{#1}
\newcommand{\SpecialCharTok}[1]{\textcolor[rgb]{0.00,0.00,0.00}{#1}}
\newcommand{\SpecialStringTok}[1]{\textcolor[rgb]{0.31,0.60,0.02}{#1}}
\newcommand{\StringTok}[1]{\textcolor[rgb]{0.31,0.60,0.02}{#1}}
\newcommand{\VariableTok}[1]{\textcolor[rgb]{0.00,0.00,0.00}{#1}}
\newcommand{\VerbatimStringTok}[1]{\textcolor[rgb]{0.31,0.60,0.02}{#1}}
\newcommand{\WarningTok}[1]{\textcolor[rgb]{0.56,0.35,0.01}{\textbf{\textit{#1}}}}
\usepackage{graphicx}
\makeatletter
\def\maxwidth{\ifdim\Gin@nat@width>\linewidth\linewidth\else\Gin@nat@width\fi}
\def\maxheight{\ifdim\Gin@nat@height>\textheight\textheight\else\Gin@nat@height\fi}
\makeatother
% Scale images if necessary, so that they will not overflow the page
% margins by default, and it is still possible to overwrite the defaults
% using explicit options in \includegraphics[width, height, ...]{}
\setkeys{Gin}{width=\maxwidth,height=\maxheight,keepaspectratio}
% Set default figure placement to htbp
\makeatletter
\def\fps@figure{htbp}
\makeatother
\setlength{\emergencystretch}{3em} % prevent overfull lines
\providecommand{\tightlist}{%
  \setlength{\itemsep}{0pt}\setlength{\parskip}{0pt}}
\setcounter{secnumdepth}{-\maxdimen} % remove section numbering
\ifluatex
  \usepackage{selnolig}  % disable illegal ligatures
\fi

\title{Untitled}
\author{Yijie Wu}
\date{5/28/2021}

\begin{document}
\maketitle

\hypertarget{question-1}{%
\section{Question 1}\label{question-1}}

\hypertarget{a.}{%
\subsection{(a).}\label{a.}}

\hypertarget{i.}{%
\subsubsection{(i).}\label{i.}}

\begin{Shaded}
\begin{Highlighting}[]
\FunctionTok{library}\NormalTok{(Matrix)}
\NormalTok{V}\OtherTok{=}\FunctionTok{Diagonal}\NormalTok{(}\AttributeTok{x =} \FunctionTok{seq}\NormalTok{(}\FloatTok{0.8}\NormalTok{,}\FloatTok{1.25}\NormalTok{,}\FloatTok{0.05}\NormalTok{))}
\NormalTok{r}\OtherTok{=}\FunctionTok{rep}\NormalTok{(}\DecValTok{0}\NormalTok{, }\DecValTok{10}\NormalTok{)}
\end{Highlighting}
\end{Shaded}

\begin{Shaded}
\begin{Highlighting}[]
\FunctionTok{set.seed}\NormalTok{(}\DecValTok{12345}\NormalTok{)}
\NormalTok{X0 }\OtherTok{\textless{}{-}}\NormalTok{ MASS}\SpecialCharTok{::}\FunctionTok{mvrnorm}\NormalTok{(}\AttributeTok{n=}\DecValTok{120}\NormalTok{, }\AttributeTok{mu =}\NormalTok{ r, }\AttributeTok{Sigma =}\NormalTok{ V)}
\end{Highlighting}
\end{Shaded}

\begin{Shaded}
\begin{Highlighting}[]
\FunctionTok{colMeans}\NormalTok{(X0)}
\end{Highlighting}
\end{Shaded}

\begin{verbatim}
##  [1]  0.06049362 -0.11921312 -0.22629882  0.16626835  0.04674217  0.01787684
##  [7]  0.06585541  0.07328721  0.03908269  0.23791583
\end{verbatim}

\begin{Shaded}
\begin{Highlighting}[]
\NormalTok{r\_mean}\OtherTok{=}\FunctionTok{colMeans}\NormalTok{(X0)}
\NormalTok{z}\OtherTok{=}\NormalTok{X0}\SpecialCharTok{{-}}\NormalTok{r\_mean}
\end{Highlighting}
\end{Shaded}

\begin{Shaded}
\begin{Highlighting}[]
\NormalTok{res}\OtherTok{=}\DecValTok{0}
\ControlFlowTok{for}\NormalTok{ (r }\ControlFlowTok{in} \DecValTok{1}\SpecialCharTok{:}\FunctionTok{nrow}\NormalTok{(z))}
\NormalTok{  res}\OtherTok{=}\NormalTok{res}\SpecialCharTok{+}\NormalTok{z[r,]}\SpecialCharTok{\%*\%} \FunctionTok{t}\NormalTok{(z[r,])}
\NormalTok{  res}\OtherTok{=}\NormalTok{res}\SpecialCharTok{/}\FunctionTok{nrow}\NormalTok{(z)}
\NormalTok{V2}\OtherTok{=}\NormalTok{res}
\end{Highlighting}
\end{Shaded}

\begin{Shaded}
\begin{Highlighting}[]
\NormalTok{e }\OtherTok{\textless{}{-}} \FunctionTok{eigen}\NormalTok{(V)}
\NormalTok{e}\SpecialCharTok{$}\NormalTok{values}
\end{Highlighting}
\end{Shaded}

\begin{verbatim}
##  [1] 1.25 1.20 1.15 1.10 1.05 1.00 0.95 0.90 0.85 0.80
\end{verbatim}

\begin{Shaded}
\begin{Highlighting}[]
\NormalTok{e2 }\OtherTok{\textless{}{-}} \FunctionTok{eigen}\NormalTok{(V2)}
\NormalTok{e2}\SpecialCharTok{$}\NormalTok{values}
\end{Highlighting}
\end{Shaded}

\begin{verbatim}
##  [1] 1.8008256 1.4972401 1.4256470 1.0413507 1.0038687 0.8939928 0.8202445
##  [8] 0.7409669 0.6428633 0.5812950
\end{verbatim}

\begin{Shaded}
\begin{Highlighting}[]
\FunctionTok{library}\NormalTok{(ggplot2)}
\FunctionTok{library}\NormalTok{(RColorBrewer)}
\NormalTok{cbPalette }\OtherTok{\textless{}{-}} \FunctionTok{brewer.pal}\NormalTok{(}\DecValTok{10}\NormalTok{, }\AttributeTok{name =} \StringTok{"Paired"}\NormalTok{)}
\FunctionTok{ggplot}\NormalTok{() }\SpecialCharTok{+}
    \FunctionTok{geom\_line}\NormalTok{(}\AttributeTok{mapping =} \FunctionTok{aes}\NormalTok{(}\AttributeTok{x =} \FunctionTok{seq}\NormalTok{(}\DecValTok{1}\NormalTok{,}\DecValTok{10}\NormalTok{,}\DecValTok{1}\NormalTok{), }\AttributeTok{y =} \FunctionTok{eigen}\NormalTok{(V)}\SpecialCharTok{$}\NormalTok{values, }\AttributeTok{color =} \StringTok{"True Covariance Matrix"}\NormalTok{)) }\SpecialCharTok{+}
    \FunctionTok{geom\_line}\NormalTok{(}\AttributeTok{mapping =} \FunctionTok{aes}\NormalTok{(}\AttributeTok{x =} \FunctionTok{seq}\NormalTok{(}\DecValTok{1}\NormalTok{,}\DecValTok{10}\NormalTok{,}\DecValTok{1}\NormalTok{), }\AttributeTok{y =} \FunctionTok{eigen}\NormalTok{(V2)}\SpecialCharTok{$}\NormalTok{values, }\AttributeTok{color =} \StringTok{"Estimated Covariance Matrix"}\NormalTok{)) }\SpecialCharTok{+}
    \FunctionTok{labs}\NormalTok{(}\AttributeTok{x =} \StringTok{"\#"}\NormalTok{, }\AttributeTok{y =} \StringTok{"EigenValues"}\NormalTok{, }\AttributeTok{title =} \StringTok{"EigenValue Chart"}\NormalTok{)}
\end{Highlighting}
\end{Shaded}

\includegraphics{q1_files/figure-latex/unnamed-chunk-8-1.pdf}

The eigenvalue is distorted when using only a finite set of observations

\hypertarget{ii.}{%
\subsubsection{(ii).}\label{ii.}}

\begin{Shaded}
\begin{Highlighting}[]
\FunctionTok{library}\NormalTok{(CVXR)}
\end{Highlighting}
\end{Shaded}

\begin{verbatim}
## 
## Attaching package: 'CVXR'
\end{verbatim}

\begin{verbatim}
## The following object is masked from 'package:stats':
## 
##     power
\end{verbatim}

\begin{Shaded}
\begin{Highlighting}[]
\NormalTok{NumAssets}\OtherTok{=}\DecValTok{10}
\NormalTok{w }\OtherTok{\textless{}{-}} \FunctionTok{Variable}\NormalTok{(NumAssets)   }\CommentTok{\# decision variables}
\NormalTok{risk }\OtherTok{\textless{}{-}} \FunctionTok{quad\_form}\NormalTok{(w, V2)  }\CommentTok{\# This is w\textquotesingle{} Sample\_Cov w}
\NormalTok{constraints }\OtherTok{\textless{}{-}} \FunctionTok{list}\NormalTok{(w }\SpecialCharTok{\textgreater{}=} \DecValTok{0}\NormalTok{, }\FunctionTok{sum}\NormalTok{(w) }\SpecialCharTok{==} \DecValTok{1}\NormalTok{)    }
\NormalTok{prob }\OtherTok{\textless{}{-}} \FunctionTok{Problem}\NormalTok{(}\FunctionTok{Minimize}\NormalTok{(risk), constraints)}
\NormalTok{result }\OtherTok{\textless{}{-}} \FunctionTok{solve}\NormalTok{(prob)}
\NormalTok{MinVar }\OtherTok{\textless{}{-}}\NormalTok{ result}\SpecialCharTok{$}\FunctionTok{getValue}\NormalTok{(risk)}
\NormalTok{w\_MinVar }\OtherTok{\textless{}{-}}\NormalTok{ result}\SpecialCharTok{$}\FunctionTok{getValue}\NormalTok{(w)}
\NormalTok{return\_Min }\OtherTok{\textless{}{-}} \FunctionTok{t}\NormalTok{(r\_mean) }\SpecialCharTok{\%*\%}\NormalTok{ w\_MinVar}
\end{Highlighting}
\end{Shaded}

\begin{Shaded}
\begin{Highlighting}[]
\NormalTok{expected}\OtherTok{=}\FunctionTok{t}\NormalTok{(w\_MinVar) }\SpecialCharTok{\%*\%}\NormalTok{V2 }\SpecialCharTok{\%*\%}\NormalTok{w\_MinVar}
\NormalTok{actual}\OtherTok{=}\FunctionTok{t}\NormalTok{(w\_MinVar) }\SpecialCharTok{\%*\%}\NormalTok{V }\SpecialCharTok{\%*\%}\NormalTok{w\_MinVar}
\end{Highlighting}
\end{Shaded}

\begin{Shaded}
\begin{Highlighting}[]
\FunctionTok{library}\NormalTok{(CVXR)}

\NormalTok{NumAssets}\OtherTok{=}\DecValTok{10}
\NormalTok{w }\OtherTok{\textless{}{-}} \FunctionTok{Variable}\NormalTok{(NumAssets)   }\CommentTok{\# decision variables}
\NormalTok{risk }\OtherTok{\textless{}{-}} \FunctionTok{quad\_form}\NormalTok{(w, V)  }\CommentTok{\# This is w\textquotesingle{} Sample\_Cov w}
\NormalTok{constraints }\OtherTok{\textless{}{-}} \FunctionTok{list}\NormalTok{(w }\SpecialCharTok{\textgreater{}=} \DecValTok{0}\NormalTok{, }\FunctionTok{sum}\NormalTok{(w) }\SpecialCharTok{==} \DecValTok{1}\NormalTok{)    }
\NormalTok{prob }\OtherTok{\textless{}{-}} \FunctionTok{Problem}\NormalTok{(}\FunctionTok{Minimize}\NormalTok{(risk), constraints)}
\NormalTok{result }\OtherTok{\textless{}{-}} \FunctionTok{solve}\NormalTok{(prob)}
\NormalTok{MinVar }\OtherTok{\textless{}{-}}\NormalTok{ result}\SpecialCharTok{$}\FunctionTok{getValue}\NormalTok{(risk)}
\NormalTok{w\_MinVar2 }\OtherTok{\textless{}{-}}\NormalTok{ result}\SpecialCharTok{$}\FunctionTok{getValue}\NormalTok{(w)}
\NormalTok{return\_Min }\OtherTok{\textless{}{-}} \FunctionTok{t}\NormalTok{(r\_mean) }\SpecialCharTok{\%*\%}\NormalTok{ w\_MinVar}
\end{Highlighting}
\end{Shaded}

\begin{Shaded}
\begin{Highlighting}[]
\NormalTok{true}\OtherTok{=}\FunctionTok{t}\NormalTok{(w\_MinVar2) }\SpecialCharTok{\%*\%}\NormalTok{V }\SpecialCharTok{\%*\%}\NormalTok{w\_MinVar2}
\NormalTok{expected}
\end{Highlighting}
\end{Shaded}

\begin{verbatim}
##           [,1]
## [1,] 0.1251664
\end{verbatim}

\begin{Shaded}
\begin{Highlighting}[]
\NormalTok{df }\OtherTok{\textless{}{-}} \FunctionTok{data.frame}\NormalTok{(}\AttributeTok{dose=}\FunctionTok{c}\NormalTok{(}\StringTok{"Estimated"}\NormalTok{, }\StringTok{"Actual"}\NormalTok{, }\StringTok{"True"}\NormalTok{),}
                \AttributeTok{len=}\FunctionTok{c}\NormalTok{(expected[}\DecValTok{1}\NormalTok{], actual[}\DecValTok{1}\NormalTok{], true[}\DecValTok{1}\NormalTok{]))}
\NormalTok{p}\OtherTok{\textless{}{-}}\FunctionTok{ggplot}\NormalTok{(}\AttributeTok{data=}\NormalTok{df, }\FunctionTok{aes}\NormalTok{(}\AttributeTok{x=}\NormalTok{dose, }\AttributeTok{y=}\NormalTok{len)) }\SpecialCharTok{+}
  \FunctionTok{geom\_bar}\NormalTok{(}\AttributeTok{stat=}\StringTok{"identity"}\NormalTok{)}
\NormalTok{p}
\end{Highlighting}
\end{Shaded}

\includegraphics{q1_files/figure-latex/unnamed-chunk-13-1.pdf}

\hypertarget{iii.}{%
\subsubsection{(iii).}\label{iii.}}

The actual variance is always larger than the true variance For the
estimated variance, it is really unstable and there is not an obvious
trend.

\hypertarget{b.}{%
\subsection{b.}\label{b.}}

\hypertarget{i.-1}{%
\subsubsection{(i).}\label{i.-1}}

\begin{Shaded}
\begin{Highlighting}[]
\NormalTok{lamda}\OtherTok{=}\FunctionTok{eigen}\NormalTok{(V2)}\SpecialCharTok{$}\NormalTok{values}
\NormalTok{lamda}
\end{Highlighting}
\end{Shaded}

\begin{verbatim}
##  [1] 1.8008256 1.4972401 1.4256470 1.0413507 1.0038687 0.8939928 0.8202445
##  [8] 0.7409669 0.6428633 0.5812950
\end{verbatim}

\begin{Shaded}
\begin{Highlighting}[]
\NormalTok{lamda\_mean}\OtherTok{=}\FunctionTok{mean}\NormalTok{(lamda)}
\end{Highlighting}
\end{Shaded}

\begin{Shaded}
\begin{Highlighting}[]
\NormalTok{C}\OtherTok{=}\NormalTok{lamda\_mean}\SpecialCharTok{*}\FunctionTok{diag}\NormalTok{(}\DecValTok{10}\NormalTok{)}
\end{Highlighting}
\end{Shaded}

\begin{Shaded}
\begin{Highlighting}[]
\FunctionTok{library}\NormalTok{(psych)}
\end{Highlighting}
\end{Shaded}

\begin{verbatim}
## 
## Attaching package: 'psych'
\end{verbatim}

\begin{verbatim}
## The following object is masked from 'package:CVXR':
## 
##     logistic
\end{verbatim}

\begin{verbatim}
## The following objects are masked from 'package:ggplot2':
## 
##     %+%, alpha
\end{verbatim}

\begin{Shaded}
\begin{Highlighting}[]
\NormalTok{res}\OtherTok{=}\DecValTok{0}
\ControlFlowTok{for}\NormalTok{ (r }\ControlFlowTok{in} \DecValTok{1}\SpecialCharTok{:}\FunctionTok{nrow}\NormalTok{(z))}
\NormalTok{  res}\OtherTok{=}\NormalTok{res}\SpecialCharTok{+}\FunctionTok{tr}\NormalTok{((z[r,]}\SpecialCharTok{\%*\%} \FunctionTok{t}\NormalTok{(z[r,])}\SpecialCharTok{{-}}\NormalTok{V2)}\SpecialCharTok{\^{}}\DecValTok{2}\NormalTok{)}
\NormalTok{  res}\OtherTok{=}\NormalTok{res}\SpecialCharTok{/}\FunctionTok{nrow}\NormalTok{(z)}
\NormalTok{alpha}\OtherTok{=}\FunctionTok{min}\NormalTok{((res}\SpecialCharTok{/}\FunctionTok{tr}\NormalTok{((V2}\SpecialCharTok{{-}}\NormalTok{C)}\SpecialCharTok{\^{}}\DecValTok{2}\NormalTok{))}\SpecialCharTok{/}\FunctionTok{nrow}\NormalTok{(z),}\DecValTok{1}\NormalTok{)}
\end{Highlighting}
\end{Shaded}

\begin{Shaded}
\begin{Highlighting}[]
\NormalTok{V3}\OtherTok{=}\NormalTok{(}\DecValTok{1}\SpecialCharTok{{-}}\NormalTok{alpha)}\SpecialCharTok{*}\NormalTok{V2}\SpecialCharTok{+}\NormalTok{alpha}\SpecialCharTok{*}\NormalTok{C}
\end{Highlighting}
\end{Shaded}

\begin{Shaded}
\begin{Highlighting}[]
\FunctionTok{eigen}\NormalTok{(V2)}\SpecialCharTok{$}\NormalTok{values}
\end{Highlighting}
\end{Shaded}

\begin{verbatim}
##  [1] 1.8008256 1.4972401 1.4256470 1.0413507 1.0038687 0.8939928 0.8202445
##  [8] 0.7409669 0.6428633 0.5812950
\end{verbatim}

\begin{Shaded}
\begin{Highlighting}[]
\FunctionTok{library}\NormalTok{(RColorBrewer)}
\NormalTok{cbPalette }\OtherTok{\textless{}{-}} \FunctionTok{brewer.pal}\NormalTok{(}\DecValTok{10}\NormalTok{, }\AttributeTok{name =} \StringTok{"Paired"}\NormalTok{)}
\FunctionTok{ggplot}\NormalTok{() }\SpecialCharTok{+}
    \FunctionTok{geom\_line}\NormalTok{(}\AttributeTok{mapping =} \FunctionTok{aes}\NormalTok{(}\AttributeTok{x =} \FunctionTok{seq}\NormalTok{(}\DecValTok{1}\NormalTok{,}\DecValTok{10}\NormalTok{,}\DecValTok{1}\NormalTok{), }\AttributeTok{y =} \FunctionTok{eigen}\NormalTok{(V)}\SpecialCharTok{$}\NormalTok{values, }\AttributeTok{colour =} \StringTok{"True Covariance Matrix"}\NormalTok{)) }\SpecialCharTok{+}
    \FunctionTok{geom\_line}\NormalTok{(}\AttributeTok{mapping =} \FunctionTok{aes}\NormalTok{(}\AttributeTok{x =} \FunctionTok{seq}\NormalTok{(}\DecValTok{1}\NormalTok{,}\DecValTok{10}\NormalTok{,}\DecValTok{1}\NormalTok{), }\AttributeTok{y =} \FunctionTok{eigen}\NormalTok{(V2)}\SpecialCharTok{$}\NormalTok{values, }\AttributeTok{colour =} \StringTok{"Estimated Covariance Matrix"}\NormalTok{)) }\SpecialCharTok{+}
    \FunctionTok{geom\_line}\NormalTok{(}\AttributeTok{mapping =} \FunctionTok{aes}\NormalTok{(}\AttributeTok{x =} \FunctionTok{seq}\NormalTok{(}\DecValTok{1}\NormalTok{,}\DecValTok{10}\NormalTok{,}\DecValTok{1}\NormalTok{), }\AttributeTok{y =} \FunctionTok{eigen}\NormalTok{(V3)}\SpecialCharTok{$}\NormalTok{values, }\AttributeTok{colour =} \StringTok{"Shrunk Covariance Matrix"}\NormalTok{)) }\SpecialCharTok{+}
    \FunctionTok{labs}\NormalTok{(}\AttributeTok{x =} \StringTok{"\#"}\NormalTok{, }\AttributeTok{y =} \StringTok{"EigenValues"}\NormalTok{, }\AttributeTok{title =} \StringTok{"EigenValue Chart"}\NormalTok{)}
\end{Highlighting}
\end{Shaded}

\includegraphics{q1_files/figure-latex/unnamed-chunk-20-1.pdf}

The eigenvalue with shrinkage method, compared with the one only using
sample data, is much closer to the actual one

\hypertarget{ii.-1}{%
\subsubsection{(ii).}\label{ii.-1}}

\begin{Shaded}
\begin{Highlighting}[]
\FunctionTok{library}\NormalTok{(CVXR)}

\NormalTok{NumAssets}\OtherTok{=}\DecValTok{10}
\NormalTok{w }\OtherTok{\textless{}{-}} \FunctionTok{Variable}\NormalTok{(NumAssets)   }\CommentTok{\# decision variables}
\NormalTok{risk }\OtherTok{\textless{}{-}} \FunctionTok{quad\_form}\NormalTok{(w, V3)  }\CommentTok{\# This is w\textquotesingle{} Sample\_Cov w}
\NormalTok{constraints }\OtherTok{\textless{}{-}} \FunctionTok{list}\NormalTok{(w }\SpecialCharTok{\textgreater{}=} \DecValTok{0}\NormalTok{, }\FunctionTok{sum}\NormalTok{(w) }\SpecialCharTok{==} \DecValTok{1}\NormalTok{)    }
\NormalTok{prob }\OtherTok{\textless{}{-}} \FunctionTok{Problem}\NormalTok{(}\FunctionTok{Minimize}\NormalTok{(risk), constraints)}
\NormalTok{result }\OtherTok{\textless{}{-}} \FunctionTok{solve}\NormalTok{(prob)}
\NormalTok{MinVar }\OtherTok{\textless{}{-}}\NormalTok{ result}\SpecialCharTok{$}\FunctionTok{getValue}\NormalTok{(risk)}
\NormalTok{w\_MinVar3 }\OtherTok{\textless{}{-}}\NormalTok{ result}\SpecialCharTok{$}\FunctionTok{getValue}\NormalTok{(w)}
\NormalTok{return\_Min }\OtherTok{\textless{}{-}} \FunctionTok{t}\NormalTok{(r\_mean) }\SpecialCharTok{\%*\%}\NormalTok{ w\_MinVar}
\end{Highlighting}
\end{Shaded}

\begin{Shaded}
\begin{Highlighting}[]
\NormalTok{expected}\OtherTok{=}\FunctionTok{t}\NormalTok{(w\_MinVar3) }\SpecialCharTok{\%*\%}\NormalTok{V3 }\SpecialCharTok{\%*\%}\NormalTok{w\_MinVar3}
\NormalTok{actual}\OtherTok{=}\FunctionTok{t}\NormalTok{(w\_MinVar3) }\SpecialCharTok{\%*\%}\NormalTok{V }\SpecialCharTok{\%*\%}\NormalTok{w\_MinVar3}
\end{Highlighting}
\end{Shaded}

\begin{Shaded}
\begin{Highlighting}[]
\FunctionTok{library}\NormalTok{(CVXR)}

\NormalTok{NumAssets}\OtherTok{=}\DecValTok{10}
\NormalTok{w }\OtherTok{\textless{}{-}} \FunctionTok{Variable}\NormalTok{(NumAssets)   }\CommentTok{\# decision variables}
\NormalTok{risk }\OtherTok{\textless{}{-}} \FunctionTok{quad\_form}\NormalTok{(w, V)  }\CommentTok{\# This is w\textquotesingle{} Sample\_Cov w}
\NormalTok{constraints }\OtherTok{\textless{}{-}} \FunctionTok{list}\NormalTok{(w }\SpecialCharTok{\textgreater{}=} \DecValTok{0}\NormalTok{, }\FunctionTok{sum}\NormalTok{(w) }\SpecialCharTok{==} \DecValTok{1}\NormalTok{)    }
\NormalTok{prob }\OtherTok{\textless{}{-}} \FunctionTok{Problem}\NormalTok{(}\FunctionTok{Minimize}\NormalTok{(risk), constraints)}
\NormalTok{result }\OtherTok{\textless{}{-}} \FunctionTok{solve}\NormalTok{(prob)}
\NormalTok{MinVar }\OtherTok{\textless{}{-}}\NormalTok{ result}\SpecialCharTok{$}\FunctionTok{getValue}\NormalTok{(risk)}
\NormalTok{w\_MinVar2 }\OtherTok{\textless{}{-}}\NormalTok{ result}\SpecialCharTok{$}\FunctionTok{getValue}\NormalTok{(w)}
\NormalTok{return\_Min }\OtherTok{\textless{}{-}} \FunctionTok{t}\NormalTok{(r\_mean) }\SpecialCharTok{\%*\%}\NormalTok{ w\_MinVar}
\end{Highlighting}
\end{Shaded}

\begin{Shaded}
\begin{Highlighting}[]
\NormalTok{true}\OtherTok{=}\FunctionTok{t}\NormalTok{(w\_MinVar2) }\SpecialCharTok{\%*\%}\NormalTok{V }\SpecialCharTok{\%*\%}\NormalTok{w\_MinVar2}
\end{Highlighting}
\end{Shaded}

\begin{Shaded}
\begin{Highlighting}[]
\NormalTok{df }\OtherTok{\textless{}{-}} \FunctionTok{data.frame}\NormalTok{(}\AttributeTok{dose=}\FunctionTok{c}\NormalTok{(}\StringTok{"Estimated"}\NormalTok{, }\StringTok{"Actual"}\NormalTok{, }\StringTok{"True"}\NormalTok{),}
                \AttributeTok{len=}\FunctionTok{c}\NormalTok{(expected[}\DecValTok{1}\NormalTok{], actual[}\DecValTok{1}\NormalTok{], true[}\DecValTok{1}\NormalTok{]))}
\NormalTok{p}\OtherTok{\textless{}{-}}\FunctionTok{ggplot}\NormalTok{(}\AttributeTok{data=}\NormalTok{df, }\FunctionTok{aes}\NormalTok{(}\AttributeTok{x=}\NormalTok{dose, }\AttributeTok{y=}\NormalTok{len)) }\SpecialCharTok{+}
  \FunctionTok{geom\_bar}\NormalTok{(}\AttributeTok{stat=}\StringTok{"identity"}\NormalTok{)}
\NormalTok{p}
\end{Highlighting}
\end{Shaded}

\includegraphics{q1_files/figure-latex/unnamed-chunk-25-1.pdf}

\hypertarget{iii.-1}{%
\subsubsection{(iii).}\label{iii.-1}}

The actual variance is much closer to the true variance. Also, comparing
to part b, now the estimated variance is less unstable.

\end{document}
